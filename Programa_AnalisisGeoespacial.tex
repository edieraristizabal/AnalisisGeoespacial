%#############################PREAMBLE#############################################
\documentclass[a4paper,twoside,11pt,]{article}

\usepackage[spanish]{babel}
\usepackage{graphicx}
\usepackage{float}
\usepackage[skins]{tcolorbox}
\usepackage{titlepic}

\usepackage{fancyhdr}
\pagestyle{fancy}
\lhead{Análisis Geoespacial}
\rhead{\thepage}
\cfoot{Programa}
\renewcommand{\headrulewidth}{0.4pt}
\renewcommand{\footrulewidth}{0.4pt}
\usepackage{hyperref} 
\usepackage{textcomp}

\graphicspath{{G:/My Drive/FIGURAS/}}

\title {PROGRAMA  CURSO\\ ANÁLISIS GEOESPACIAL}
\author{Prof.: Edier Aristizábal\\[5ex]
\includegraphics[width=10.0cm]{unal2.png}
}
\date{}

%################################BODY############################################
\begin{document}
\maketitle

\emph {versión}: \today

\emph {Classroom code:} 32cjlau

\section* {Introducción}
El curso Análisis Geoespacial está orientado para estudiantes de posgrados que deseen adquirir conocimientos sobre sensores remotos y datos geoespaciales en un contexto ambiental, utilizando herramientas tipo Sistemas de Información Geográfica (SIG), Google Earth Engine (GEE), Big Data, y programación en lenguaje Python.\\
El curso es teórico - práctico. Se dictarán clases teóricas con las técnicas y modelos a utilizar, y clases prácticas donde se resolverán dudas con el manejo de las herramientas. El curso se evaluará a través de un trabajo individual durante todo el curso, donde el estudiante implementará en una cuenca de su elección las herramientas de análisis presentadas en el curso.

\section{PROGRAMA}
El contenido del curso comprende los siguientes temas a desarrollar:\\

\subsection*{Introducción al curso}

\subsection {Ambiente de trabajo}
QGIS. GEE. Jupyter Lab. Google Colab.

\subsection {Sensores Remotos}
Radiación electromagnética. Sensores ópticos. Sensores de antena.

\begin{tcolorbox}[enhanced,width=5in,center upper,  fontupper=\large\bfseries,drop shadow southwest,sharp corners]
Taller 1 (20\%)--Presentacion 
\end{tcolorbox}

\subsection {Sistemas de Informacion Geografica --GIS--}
Datos espaciales. Formatos. Correcciones radiométricas.  Tratamiento de imágenes de satélite. Evaluación.
\begin{tcolorbox}[enhanced,width=5in,center upper,  fontupper=\large\bfseries,drop shadow southwest,sharp corners]
Taller 2 (20\%)--GIS
\end{tcolorbox}

\subsection {Google Earth Engine}
Explorer. Code Editor. Landsat. Sentinel. MODIS. CHIRPS. TerraClimate. SMAP. SRTM. ALOS.
\begin{tcolorbox}[enhanced,width=5in,center upper,  fontupper=\large\bfseries,drop shadow southwest,sharp corners]
Taller 3 (20\%)--GEE Explorer
\end{tcolorbox}
\begin{tcolorbox}[enhanced,width=5in,center upper,  fontupper=\large\bfseries,drop shadow southwest,sharp corners]
Taller 4 (20\%)--GEE Code
\end{tcolorbox}

\subsection {Python}
Notebooks en Google Colab. \& Jupyter Lab. GEE en QGIS. GDAL. Rasterio.
\begin{tcolorbox}[enhanced,width=5in,center upper,  fontupper=\large\bfseries,drop shadow southwest,sharp corners]
Taller 5 (20\%)--Trabajo final
\end{tcolorbox}

\section{EVALUACIÓN DEL CURSO}
El curso se evaluará con 5 talleres, con un valor individual de 20\%. El primero de ellos es en grupo, y lo 4 talleres restantes individuales. Las guías y descripción de los talleres se encuentran en el curso digital. No se aceptan entregas por correo electrónico, todas las entregas serán a través de la plataforma de Classroom.

\end{document}
