%###########################PRESENTACION##########################################
%Modo presentación
\documentclass[14pt]{beamer}

%Modo handout
%\documentclass[handout,compress]{beamer}
%\usepackage{pgfpages}
%\pgfpagesuselayout{4 on 1}[border shrink=1mm]

\usepackage{graphicx,pstricks}
\usepackage{beamerthemeCambridgeUS}
\usepackage{subfig}
\usepackage{tikz}
\usepackage{amsmath}
\usepackage{hyperref}

\graphicspath{{G:/My Drive/FIGURAS/}}
\setbeamercovered{transparent}

\title[Tratamiento de Imágenes]{ANÁLISIS GEOESPACIAL}
\author[Edier Aristizábal]{Edier V. Aristizábal G.}
\institute{\emph{evaristizabalg@unal.edu.co}}
\date{\tiny{(Versión:\today)}}
\usepackage{textpos}

\addtobeamertemplate{headline}{}{%
	\begin{textblock*}{2mm}(.9\textwidth,0cm)
	\hfill\includegraphics[height=1cm]{un}
	\end{textblock*}
			}
%############################INICIO#############################################
\begin{document}
\begin{frame}
\titlepage
\centering
	\includegraphics[width=5cm]{unal}\hspace*{4.75cm}~%
   	\includegraphics[width=2cm]{logo3}
\end{frame}
%#############################SLIDE
\begin{frame}
\frametitle{Tratamiento de imágenes de satélite}
%\framesubtitle{}
\scriptsize{Existen una gran cantidad de procedimientos para el análisis de imágenes de satélite. En este curso nos concentraremos en 4 de ellas:}
\begin{itemize}
\item Pro-procesamiento de imágenes
\item Mejoramiento de imágenes
\item Transformaciones de imágenes
\item Clasificación de imágenes.
\end{itemize}
  \begin{figure}
    \centering
    \includegraphics[height=.4\textheight]{band}
    %\caption{This is the caption.}
  \end{figure}
\end{frame}
%################################SLIDE
\begin{frame}
%\framesubtitle{}
\small{Cualquier imagen adquirida por un sensor remoto presenta una serie de alteraciones radiométricas y geométricas debidas a factores como:}
  \begin{figure}
    \centering
    \includegraphics[height=.7\textheight]{correciones}
    %\caption{This is the caption.}
  \end{figure}
\end{frame}
%################################SLIDE
\begin{frame}
\scriptsize{}
  \begin{figure}
    \centering
    \includegraphics[height=.8\textheight]{calibracion}
    %\caption{This is the caption.}
  \end{figure}
\end{frame}
%################################SLIDE
\begin{frame}
\frametitle{Striping}
\scriptsize{}
  \begin{figure}
    \centering
    \includegraphics[height=.7\textheight]{striping}
    %\caption{This is the caption.}
  \end{figure}
\end{frame}
%################################SLIDE
\begin{frame}
\frametitle{Line Drop}
\scriptsize{}
  \begin{figure}
    \centering
    \includegraphics[height=.7\textheight]{linedrop}
    %\caption{This is the caption.}
  \end{figure}
\end{frame}
%################################SLIDE
\begin{frame}
\frametitle{Bit Error}
\scriptsize{}
  \begin{figure}
    \centering
    \includegraphics[height=.6\textheight]{biterror}
    %\caption{This is the caption.}
  \end{figure}
\end{frame}
%################################SLIDE
\begin{frame}
\scriptsize{}
  \begin{figure}
    \centering
    \includegraphics[height=.7\textheight]{toa}
     \end{figure}
\end{frame}
%################################SLIDE
\begin{frame}
\frametitle{Corrección \& Calibración}
\scriptsize{}
  \begin{figure}
    \centering
    \includegraphics[height=.7\textheight]{calibracion2}
     \end{figure}
\end{frame}
%################################SLIDE
\begin{frame}
\scriptsize{}
  \begin{figure}
    \centering
    \includegraphics[height=.7\textheight]{toa2}
     \end{figure}
\end{frame}
%################################SLIDE
\begin{frame}
\frametitle{Conversión a Radiancia espectral TOA}
\scriptsize{}
  \begin{figure}
    \centering
    \includegraphics[height=.6\textheight]{convertoa}
    %\caption{This is the caption.}
  \end{figure}
\end{frame}
%################################SLIDE
\begin{frame}
\frametitle{Conversión a Reflectividad TOA}
\scriptsize{}
  \begin{figure}
    \centering
    \includegraphics[height=.8\textheight]{converrefli}
    %\caption{This is the caption.}
  \end{figure}
\end{frame}
%################################SLIDE
\begin{frame}
\scriptsize{}
  \begin{figure}
    \centering
    \includegraphics[height=.8\textheight]{boa}
    %\caption{This is the caption.}
  \end{figure}
\end{frame}
%################################SLIDE
\begin{frame}
\frametitle{Radiación Termal}
\small{La Temperatura cinética es la manifestación interna de la energía traslacional promedio de la moléculas que componen un cuerpo (temperatura cinética). Como consecuencia  los objetos irradian energía en función de su temperatura (Temperatura radiante), adicionalmente esta temperatura sensada es de los primeros $50$ cm, puede no ser representativa de todo el objeto.\\
Sin embargo debido a la diferencia de emisividad que tienen los objetos, un cuerpo puede tener la misma temperatura y aun así tener diferente radiancia. Solo los cuerpo negros presentan que la Trad $=$ Tcin, para los demas cuerpos la temperatura radiante siempre es menor, ya que la emisividad es menor que $1$.}
\begin{figure}
    \centering
    \includegraphics[height=.3\textheight]{termal}
    %\caption{This is the caption.}
  \end{figure}
\end{frame}
%################################SLIDE
\begin{frame}
\frametitle{Conversión a T de brillo }
  \begin{figure}
    \centering
    \includegraphics[height=.6\textheight]{brillo}
    %\caption{This is the caption.}
  \end{figure}
\end{frame}
%################################SLIDE
\begin{frame}
\frametitle{Ortorectificación}
\scriptsize{}
  \begin{figure}
    \centering
    \includegraphics[height=.8\textheight]{proyeorto}
    %\caption{This is the caption.}
  \end{figure}
\end{frame}
%################################SLIDE
\begin{frame}
\scriptsize{}
  \begin{figure}
    \centering
    \includegraphics[height=.8\textheight]{ortoejemplo}
    %\caption{This is the caption.}
  \end{figure}
\end{frame}
%################################SLIDE
\begin{frame}
\scriptsize{}
  \begin{figure}
    \centering
    \includegraphics[height=.8\textheight]{referen}
    %\caption{This is the caption.}
  \end{figure}
\end{frame}
%################################SLIDE
\begin{frame}
  \begin{figure}
    \centering
    \subfloat[Directa]{\includegraphics[width=5cm]{interp1}}\qquad
    \subfloat[Interpolación bilineal]{\includegraphics[width=4cm]{interp2}}
    \label{fig:1}
  \end{figure}
  \begin{figure}
    \centering
    \subfloat[Interpolación cúbica]{\includegraphics[width=3cm]{interp3}}\qquad
    \subfloat[Vecino mas Cercano]{\includegraphics[width=3cm]{interp4}}
    \label{fig:2}
  \end{figure}
\end{frame}
%############################SLIDE
\begin{frame}
\scriptsize{}
  \begin{figure}
    \centering
    \includegraphics[height=.8\textheight]{orto2}
    %\caption{This is the caption.}
  \end{figure}
\end{frame}
%################################SLIDE
\begin{frame}
\scriptsize{}
  \begin{figure}
    \centering
    \includegraphics[height=.8\textheight]{interp5}
    %\caption{This is the caption.}
  \end{figure}
\end{frame}
%################################SLIDE
\begin{frame}
\scriptsize{}
  \begin{figure}
    \centering
    \includegraphics[height=.8\textheight]{orto}
    %\caption{This is the caption.}
  \end{figure}
\end{frame}
%################################SLIDE
\begin{frame}
\frametitle{Mejoramiento de Imágenes}
  \begin{figure}
    \centering
    \includegraphics[height=.8\textheight]{hist}
    %\caption{This is the caption.}
  \end{figure}
\tiny{}
\end{frame}
%################################SLIDE
\begin{frame}
\frametitle{Ajustes del Histograma}
\framesubtitle{\emph{Strech}}
  \begin{figure}
    \centering
    \includegraphics[height=.8\textheight]{strech}
    %\caption{This is the caption.}
  \end{figure}
\tiny{}
\end{frame}
%################################SLIDE
\begin{frame}
\frametitle{Ajustes del Histograma}
\framesubtitle{\emph{Strech}}
  \begin{figure}
    \centering
    \includegraphics[height=.8\textheight]{strech2}
    %\caption{This is the caption.}
  \end{figure}
\tiny{}
\end{frame}
%################################SLIDE
\begin{frame}
  \begin{figure}
    \centering
    \includegraphics[height=.8\textheight]{hist2}
    %\caption{This is the caption.}
  \end{figure}
\tiny{}
\end{frame}
%################################SLIDE
\begin{frame}
\frametitle{Filtros}
  \begin{figure}
    \centering
    \includegraphics[height=.8\textheight]{filtros}
    %\caption{This is the caption.}
  \end{figure}
\tiny{}
\end{frame}
%################################SLIDE
\begin{frame}
  \begin{figure}
    \centering
    \includegraphics[height=.8\textheight]{highlowpass}
    %\caption{This is the caption.}
  \end{figure}
\tiny{}
\end{frame}
%################################SLIDE
\begin{frame}
  \begin{figure}
    \centering
    \includegraphics[height=.5\textheight]{filtrodir}
    %\caption{This is the caption.}
  \end{figure}
\tiny{}
\end{frame}
%################################SLIDE
\begin{frame}
  \begin{figure}
    \centering
    \includegraphics[height=.8\textheight]{filtrodir2}
    %\caption{This is the caption.}
  \end{figure}
\tiny{}
\end{frame}
%################################SLIDE
\begin{frame}
  \begin{figure}
    \centering
    \includegraphics[height=.8\textheight]{filtro2}
    %\caption{This is the caption.}
  \end{figure}
\tiny{}
\end{frame}
%################################SLIDE
\begin{frame}
\frametitle{Cociente}
  \begin{figure}
    \centering
    \includegraphics[height=.8\textheight]{cociente}
    %\caption{This is the caption.}
  \end{figure}
\tiny{}
\end{frame}
%################################SLIDE
\begin{frame}
\frametitle{Combinación de bandas}
  \begin{figure}
    \centering
    \includegraphics[height=.8\textheight]{ihs}
    %\caption{This is the caption.}
  \end{figure}
\tiny{}
\end{frame}
%################################SLIDE
\begin{frame}
  \begin{figure}
    \centering
    \includegraphics[height=.8\textheight]{compo}
    %\caption{This is the caption.}
  \end{figure}
\tiny{}
\end{frame}
%################################SLIDE
\begin{frame}
  \begin{figure}
    \centering
    \includegraphics[height=.8\textheight]{compo2}
    %\caption{This is the caption.}
  \end{figure}
\tiny{}
\end{frame}
%################################SLIDE
\begin{frame}
\frametitle{Combinación de bandas y cociente}
  \begin{figure}
    \centering
    \includegraphics[height=.8\textheight]{compcocientes}
  \end{figure}
\end{frame}
%################################SLIDE
\begin{frame}
  \begin{figure}
    \centering
    \includegraphics[height=.8\textheight]{number}
    %\caption{This is the caption.}
  \end{figure}
\tiny{}
\end{frame}
%################################SLIDE
\begin{frame}
\frametitle{Transformación de imágenes}
\framesubtitle{Componentes Principales}
  \begin{figure}
    \centering
    \includegraphics[height=.8\textheight]{pc}
    %\caption{This is the caption.}
  \end{figure}
\tiny{}
\end{frame}
%################################SLIDE
\begin{frame}
  \begin{figure}
    \centering
    \includegraphics[height=.8\textheight]{cova}
    %\caption{This is the caption.}
  \end{figure}
\tiny{}
\end{frame}
%################################SLIDE
\begin{frame}
\frametitle{\emph{Tasselled cap}}
  \begin{figure}
    \centering
    \includegraphics[height=.7\textheight]{cap2}
     \includegraphics[height=.2\textheight]{cap3}
    %\caption{This is the caption.}
  \end{figure}
\tiny{}
\end{frame}
%################################SLIDE
\begin{frame}
  \begin{figure}
    \centering
    \includegraphics[height=.7\textheight]{cap}
    %\caption{This is the caption.}
  \end{figure}
\tiny{}
\end{frame}
%################################SLIDE
\begin{frame}
\frametitle{Transformación RGB - IHS}
  \begin{figure}
    \centering
    \includegraphics[height=.8\textheight]{ihs2}
    %\caption{This is the caption.}
  \end{figure}
\tiny{}
\end{frame}
%%%%%%%%%%%%%%%%%%%%%%%%%%%%%%%%%%%%%%%%%%%%%%%%%%%%%%%%%%%%%%%%
\end{document}