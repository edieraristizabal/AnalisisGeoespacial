\documentclass[a4paper,twoside,11pt,]{article}

\usepackage[spanish]{babel}
\usepackage{graphicx}
\usepackage{float}
\usepackage[skins]{tcolorbox}
\usepackage{titlepic}

\usepackage{fancyhdr}
\pagestyle{fancy}
\lhead{Análisis Geoespacial}
\rhead{\thepage}
\cfoot{Programa}
\renewcommand{\headrulewidth}{0.4pt}
\renewcommand{\footrulewidth}{0.4pt}
\usepackage{hyperref} 
\usepackage{textcomp}

\graphicspath{{G:/My Drive/FIGURAS/}}

\title {PROGRAMA  CURSO\\ ANÁLISIS GEOESPACIAL}
\author{Prof.: Edier Aristizábal\\[5ex]
\includegraphics[width=10.0cm]{unal2.png}
}
\date{}

%################################BODY############################################
\begin{document}
\maketitle

\emph {versión}: \today

\emph {Google Classroom code:} 32cjlau

\section* {Introducción}
El curso Análisis Geoespacial está orientado para estudiantes de posgrados que deseen adquirir conocimientos sobre análisis de datos geoespaciales en un contexto ambiental, utilizando herramientas tipo Sistemas de Información Geográfica (SIG), Google Earth Engine (GEE), Big Data, y programación.\\
El curso es teórico - práctico. Se dictarán clases teóricas con las técnicas y modelos a utilizar, y clases prácticas donde se utilizarán las herramientas de análisis. 

\section{PROGRAMA}
El contenido del curso comprende los siguientes temas a desarrollar:

\subsection*{Introducción al curso}

\subsection {Geospatial Computational Environment}
\begin{itemize}
    \item Python
   \begin{itemize}
    \item Jupyter Notebook - Markdown
    \item Google colaborative Notebook
    \item QGIS
   \end{itemize}
    \item Javascript
    \begin{itemize}
        \item Google Earth Engine
        \item Sentinel Hub
    \end{itemize}
    \item HTML - CSS
\end{itemize}

\subsection{Intro to Remote Sensing - GIS}
\begin{itemize}
    \item Sensores \& Resoluciones
    \item Tratamiento de imágenes
    \item QGIS
    \item Fotointepretación asistida por computador
    \item Morfometría
\end{itemize}

\subsection{Geospatial Data}
\begin{itemize}
    \item Download
    \item Data
    \item Geospatial Data
    \item Geolocation
\end{itemize}

\subsection{Geovisualization}
\begin{itemize}
    \item Plot maps
    \item GeoPandas
    \item Mapping
\end{itemize}

\subsection{Spatial analysis}
\begin{itemize}
    \item Exploratory Spatial Data Analysis (ESDA)
    \item Choropleth Mapping
    \item Spatial weights
    \item Spatial autocorrelation
\end{itemize}

\subsection{Raster analysis}
\begin{itemize}
    \item Raster
    \item Rasterio
    \item GEE
    \item Sentinel Hub
\end{itemize}

\subsection{Machine learning}
\begin{itemize}
    \item Point pattern
    \item Clustering
    \item Spatial regression
\end{itemize}

\subsection{Web mapping}
\begin{itemize}
    \item MongoDB
    \item Mapbox
    \item Carto
\end{itemize}

\section{Evaluación}
El curso se evaluará a través de un trabajo individual durante todo el curso, donde el estudiante implementará en una área de 
su elección las herramientas de análisis presentadas en el curso. Para el seguimiento se realizarán por cada estudiante tres 
presentaciones con el avance de su trabajo de la siguiente forma:

\subsection{Presentación del problema}
\begin{itemize}
    \item Porcentaje de evaluación: 20\%
    \item Tiempo: 5 min.
    \item Alcance: Presentacion del problema de investigación y fuente de información.
    \item Fecha: luego de terminar el modulo 3. Geospatial data
\end{itemize}

\subsection{Avances}
\begin{itemize}
    \item Porcentaje de evaluación: 20\%
    \item Tiempo: 10 min.
    \item Alcance: Avances.
    \item Fecha: luego de terminar el modulo 6. Raster analysis
\end{itemize}

\subsection{Presentación final}
\begin{itemize}
    \item Porcentaje de evaluación: 30\%
    \item Tiempo: 15 min.
    \item Alcance: Presentación trabajo final.
    \item Fecha: Al final del curso
\end{itemize}

El 30\% final corresponde al trabajo escrito en formato artículo (Introducción, Datos \& metodología, Resultados, Discusión, 
y Conclusiones). La entrega de este trabajo se realizará en formato PDF mediante la herramienta de Google classroom, 8 dias posterior a la 
presentación final.

\end{document}
